\documentclass{article}
\usepackage[utf8]{inputenc}

\usepackage[spanish,es-noindentfirst]{babel}
\usepackage{graphicx}
%Code Highlighting
\usepackage{minted}                 
\usepackage{lastpage}

\usepackage{biblatex}
\addbibresource{bibliografia.bib}

%Encabezados y pie de página
\usepackage{fancyhdr}               
\pagestyle{fancy}
\fancyhf{}
\fancyhead[L]{<<TÍTULO>>}
\fancyhead[R]{<<ASIGNATURA>>}
\fancyfoot[L]{Grado en Ingeniería Informática \\ Escuela de Ingeniería y Arquitectura}
\fancyfoot[R]{\thepage \hspace{1pt} de \pageref{LastPage}}

\renewcommand{\headrulewidth}{2pt}
\renewcommand{\footrulewidth}{1pt}

\usepackage{hyperref}
\hypersetup{
    colorlinks=true,
    linkcolor=blue,
    filecolor=magenta,      
    urlcolor=cyan,
    pdftitle={<<TÍTULO>>},                   
    bookmarks=true,
}
\urlstyle{same}

\title{<<TÍTULO>>}                            
\author{Javier Herrer Torres}      
\date{<<FECHA>>}                            

\begin{document}

\begin{titlepage}
    \begin{center}
        \vspace*{1cm}
            
        \Huge
        \textbf{<<TÍTULO>>}
            
        \vspace{0.5cm}
        \LARGE
        Subtítulo
            
        \vspace{1.5cm}
            
        \textbf{Javier Herrer Torres} (NIP: 776609)
            
        \vfill
            
        <<ASIGNATURA>>\\
        Grado en Ingeniería Informática\\
            
        \vspace{1.5cm}
            
        \includegraphics[width=0.6\textwidth]{eina.pdf}
            
        \vspace{1.5cm}
            
        \Large
        Escuela de Ingeniería y Arquitectura\\
        Universidad de Zaragoza\\
        Curso 2020/2021
            
    \end{center}
\end{titlepage}

\begin{abstract}
    % Resumen: es un apartado fundamental. Es lo primero que se lee y muchas veces lo único que lee. Es la síntesis de todo el trabajo realizado, qué, cómo y por qué hemos hecho el trabajo. Debe ser auto contenido y debemos esbozar nuestras conclusiones.
\end{abstract}

\tableofcontents

\section{Introducción}
% Introducción: enmarca y sitúa el trabajo a realizar.

\section{Objetivos}
% Objetivos: explica qué se quiere conseguir.

\section{Metodología}
% Describe los pasos realizados para llegar hasta los resultados. Todas aquellas decisiones de diseño tomadas en el proceso deben incluirse.

\section{Resultados}
% Resultado(s): hay que presentar los resultados, explicarlos y analizarlos.

\section{Conclusiones}
% Conclusiones: es lo último que se lee, por tanto es una sección muy importante que se debe utilizar para remarcar los mensajes que queremos que el lector reciba. Por ejemplo, si estamos evaluando un producto podemos enfatizar sus puntos fuertes y sus puntos débiles, y señalar posibilidades de mejora.

\printbibliography
%Normalmente al final se incluyen referencias, bibliografía, índice de expresiones técnicas y anexos.

\end{document}
