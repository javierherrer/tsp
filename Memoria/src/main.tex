\documentclass{article}
\usepackage[utf8]{inputenc}

\usepackage[spanish,es-noindentfirst]{babel}
\usepackage{graphicx}
%Code Highlighting
%\usepackage{minted}                 
\usepackage{lastpage}
\usepackage{array}


\usepackage{biblatex}
\addbibresource{bibliografia.bib}

%Encabezados y pie de página
\usepackage{fancyhdr}               
\pagestyle{fancy}
\fancyhf{}
\fancyhead[L]{Práctica de la asignatura}
\fancyhead[R]{Algoritmia básica}
\fancyfoot[L]{Grado en Ingeniería Informática \\ Escuela de Ingeniería y Arquitectura}
\fancyfoot[R]{\thepage \hspace{1pt} de \pageref{LastPage}}

\renewcommand{\headrulewidth}{2pt}
\renewcommand{\footrulewidth}{1pt}

\usepackage{hyperref}
\usepackage[T1]{fontenc}
\hypersetup{
    colorlinks=true,
    linkcolor=blue,
    filecolor=magenta,      
    urlcolor=cyan,
    pdftitle={Práctica de la asignatura: El problema del viajante de comercio},                   
    bookmarks=true,
}
\urlstyle{same}

\title{Práctica de la asignatura: El problema del viajante de comercio}                            
\author{Javier Herrer Torres}      
\date{Abril 2021}                            

\begin{document}

\begin{titlepage}
    \begin{center}
        \vspace*{1cm}
            
        \Huge
        \textbf{Práctica de la asignatura}
            
        \vspace{0.5cm}
        \LARGE
        El problema del viajante de comercio
            
        \vspace{1.5cm}
            
        \textbf{Javier Herrer Torres} (NIP: 776609)\\
        \textbf{Javier Fuster Trallero} (NIP: 626901)
            
        \vfill
            
        Algoritmia básica\\
        Grado en Ingeniería Informática\\
            
        \vspace{1.5cm}
            
        \includegraphics[width=0.6\textwidth]{../images/eina}
            
        \vspace{1.5cm}
            
        \Large
        Escuela de Ingeniería y Arquitectura\\
        Universidad de Zaragoza\\
        Curso 2020/2021
            
    \end{center}
\end{titlepage}

\section{Objetivos}
% Objetivos: explica qué se quiere conseguir.
El objetivo de la práctica es implementar y comparar la eficiencia en tiempo de
distintos esquemas algorítmicos para la resolución del problema del viajante de
comercio (TSP, \textit{Travelling Salesman Problem}).

\section{Metodología}

\subsection{Fuerza bruta}
El algoritmo de \emph{fuerza bruta} para resolver el problema consiste en
intentar todas las posibilidades, es decir, calcular las longitudes de todos
los recorridos posibles, y seleccionar la de longitud mínima.
Obviamente, el coste del algoritmo crece exponencialmente con el número de
puntos a visitar.



\subsection{Algoritmo voraz}
La heurística voraz consiste en ir seleccionando parejas de puntos que serán
visitados de forma consecutiva.
Se seleccionará primero aquella pareja de puntos enrte los que la distancia
sea mínima.
A continuación, se selecciona la siguiente pareja con una distancia mínima
siempre que esta nueva elección no haga que se visite un punto dos veces o más
(es decir, que el punto aparezca tres o más veces en las parejas de puntos
seleccionadas), o se cierre un recorrido antes de haber visitado todos los
puntos.

De esta forma, si hay que visitar $n$ puntos (incluido el origen), se selecciona
un conjunto de $n$ parejas de puntos (que serán visitados de forma consecutiva)
y la solución consiste en reordenar todas esas parejas de forma que constituyan
un recorrido.

\subsection{Programación dinámica}

\begin{equation}
    g(i,S) = min_{j \in S} \{ L_{ij} + g(j, S \setminus \{ j \}  ) \}
    \caption{Función parametrizada}
\end{equation}

\subsection{Ramificación y poda}


\section{Resultados}
% Resultado(s): hay que presentar los resultados, explicarlos y analizarlos.

\begin{center}
    \begin{tabular}{||m{2.5cm} | m{2cm} | m{2cm} | m{2cm} | m{2cm}||}
        \hline
                            & Fuerza bruta   & Programación dinámica    & Ramificación y poda   \\ [0.5ex]
        \hline\hline
        Coste en tiempo     & $n!$           & $n^2 2^n$                & $n^2 2^n$              \\
        \hline
        Coste en espacio    & TODO           & $n2^n$                   & TODO                  \\ [1ex]
        \hline
   \end{tabular}
\end{center}



\section{Conclusiones}
% Conclusiones: es lo último que se lee, por tanto es una sección muy importante que se debe utilizar para remarcar los mensajes que queremos que el lector reciba. Por ejemplo, si estamos evaluando un producto podemos enfatizar sus puntos fuertes y sus puntos débiles, y señalar posibilidades de mejora.

%\printbibliography
%Normalmente al final se incluyen referencias, bibliografía, índice de expresiones técnicas y anexos.

\end{document}
