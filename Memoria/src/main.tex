\documentclass{article}
\usepackage[utf8]{inputenc}

\usepackage[spanish,es-noindentfirst]{babel}
\usepackage{graphicx}
%Code Highlighting
%\usepackage{minted}                 
\usepackage{lastpage}
\usepackage{array}


\usepackage{biblatex}

%Encabezados y pie de página
\usepackage{fancyhdr}               
\pagestyle{fancy}
\fancyhf{}
\fancyhead[L]{Práctica de la asignatura}
\fancyhead[R]{Algoritmia básica}
\fancyfoot[L]{Grado en Ingeniería Informática \\ Escuela de Ingeniería y Arquitectura}
\fancyfoot[R]{\thepage \hspace{1pt} de \pageref{LastPage}}

\renewcommand{\headrulewidth}{2pt}
\renewcommand{\footrulewidth}{1pt}

\usepackage{hyperref}
\usepackage[T1]{fontenc}
\hypersetup{
    colorlinks=true,
    linkcolor=blue,
    filecolor=magenta,      
    urlcolor=cyan,
    pdftitle={Práctica de la asignatura: El problema del viajante de comercio},                   
    bookmarks=true,
}
\urlstyle{same}

\title{Práctica de la asignatura: El problema del viajante de comercio}                            
\author{Javier Herrer Torres}      
\date{Abril 2021}                            

\begin{document}

\begin{titlepage}
    \begin{center}
        \vspace*{1cm}
            
        \Huge
        \textbf{Práctica de la asignatura}
            
        \vspace{0.5cm}
        \LARGE
        El problema del viajante de comercio
            
        \vspace{1.5cm}
            
        \textbf{Javier Herrer Torres} (NIP: 776609)\\
        \textbf{Javier Fuster Trallero} (NIP: 626901)
            
        \vfill
            
        Algoritmia básica\\
        Grado en Ingeniería Informática\\
            
        \vspace{1.5cm}
            
        \includegraphics[width=0.6\textwidth]{../images/eina}
            
        \vspace{1.5cm}
            
        \Large
        Escuela de Ingeniería y Arquitectura\\
        Universidad de Zaragoza\\
        Curso 2020/2021
            
    \end{center}
\end{titlepage}

\section{Objetivos}
% Objetivos: explica qué se quiere conseguir.
El objetivo de la práctica es implementar y comparar la eficiencia en tiempo de
distintos esquemas algorítmicos para la resolución del problema del viajante de
comercio (TSP, \textit{Travelling Salesman Problem}).

\section{Resultados}
% Resultado(s): hay que presentar los resultados, explicarlos y analizarlos.


\section{Conclusiones}
% Conclusiones: es lo último que se lee, por tanto es una sección muy importante que se debe utilizar para remarcar los mensajes que queremos que el lector reciba. Por ejemplo, si estamos evaluando un producto podemos enfatizar sus puntos fuertes y sus puntos débiles, y señalar posibilidades de mejora.

%\printbibliography
%Normalmente al final se incluyen referencias, bibliografía, índice de expresiones técnicas y anexos.

\end{document}
